\documentclass {article}

\begin{document}

\title{EXAMINATION MALPRACTICE IN
MAKERERE UNIVERSITY
}
\author{KOMAKECH RONALD 15/u/6690/EVE 215011576}

\maketitle

\section{Introduction}
I was recently asked  to submit a short report on a topic of my choice which is as stated in the subject above. Having chosen such a topic of major concern to the university and which revolves around mainly the student fraternity, I had to embark on the task of interacting with a good number of my fellow students from different colleges so as to get to the root causes of such a vice.
\subsection{Observation}
I would say that my research was a success as students willingly gave out their opinions and some went as far as describing their personal experiences regarding the topic of discussion. Below are the reasons that came up as to what has led to the increased examination malpractice in Makerere University; Dodging of lectures by some students always came up first among the students I interacted with although each had a different opinion as to what would make one dodge lectures. One student personally confessed that he never attended lectures his entire first year at the university because of the perception among students that First year is always easy and it always turns out the opposite. Another student had the opinion that First year is for having fun hence they resort to betting, going out for parties and in most cases these students are the culprits of examination malpractice. Incomplete syllabi in some course units by the time of exams makes students feel unprepared to sit for the exam of such course units hence they try as much as possible to sneak into the examination room with summaries, while some resort to writing on their thighs an act which is common among the feminine gender. One of the students who prefers their name be kept anonymous went as personal as describing the challenges he goes through so to raise tuition. This makes it difficult for him to attend lectures or even do coursework and has to pay a small portion of his salary to classmates so that his coursework gets done in time. An intriguing point which would only be backed by fellow students offering the course unit and maybe the results sheet but was quickly disputed by lecturers of the mentioned course unit also came up as a student took no time in stating that some hard course units namely;
Data Structures and Algorithms made her exchange question papers full of summaries of the solutions with her friend seated just in front of her something which underestimates her Christian virtues. Lack of serious supervision in some examination rooms was highlighted by a student in one of the colleges, who stated that the weak supervision was as a result of the large number of students enrolled for a certain course unit and the many examination rooms allocated for the course unit. This means not all rooms will have enough invigilators resulting into malpractice Abrupt tests given by some lecturers often find students with limited knowledge about the contents of that particular course unit as some of the students only resort to intensive reading after the examination schedule is out. Persistent occurrence of strikes at least once every semester disrupts the teaching schedule as well as reading time for the students. Students from Colleges that still rely on hand outs from lecturers instead of soft copy notes sent to students emails are the main victims in such a case. And during the recent strike that led to the abrupt closure of the university, students went home with only half of their respective syllabi done and no and outs to read during the holiday. It was unfortunate for them to be given work of two months to be read in one month and after sit for exams. It is therefore not a coincidence that some students were caught in the act of malpractice. Some parents have high expectations for their children and this often exerts a lot of pressure on such students as they never want to be looked at as disappointments. A student have me his own experience of how he had never got a retake from the time he joined campus and his father always used him as an example to challenge the rest of the family members. But in the final year of his study, he lost concentration for
reasons only known to him and his only saving grace was to pay another student to sit for him the examination as he did not want to disappoint his father.

\subsection{Recommendations}
Eradicating examination malpractice as a bad vice among students is almost close to an impossible task and based on the causes I gathered from my fellow students, I would recommend the following solutions; Colleges with large numbers of students pursuing a given course should be given priority when allocating examination rooms so that all students might fit in the same room especially in situations where invigilators are not enough. Lecturers should allocate some percentage of coursework marks to class attendance as this would forcefully make the students attend lectures. Roll calls should also be conducted at the being as well as at the end of every lecture as this would encourage students come for lectures and also keep up to date with what is taking place in class.
This ensures that students are not caught off guard in case of an abrupt test.
Colleges should encourage lecturers to resort to the system of giving notes for the whole semester to students as it is being done in the college of computing and information sciences. This ensures that students can continue reading their notes even in the incidence of a strike. Lecturers should always follow the examination schedules set by the various colleges or engage into dialogue with their students before organizing tests that are out of the schedule. This allows students especially those who are working enough time to prepare for the tests and this eliminates such negative thoughts of malpractice from their heads. Issues that are of major concern such as lecturers’ salaries, salaries of the non-teaching stuff should be addressed before the start of the academic semester so that both lecturers and students concentrate on their main roles at the university. This would ensure that syllabi are completed before exams commence such that the rest of the time is used for revision.

\section{Conclusion}
All in all examination malpractice is such an evil that both the administration and the students fraternity should work hand in hand to help eradicate the vice as we struggle to build a future free from such evils.
















\end{document}